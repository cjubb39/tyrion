\documentclass[twocolumn]{article}

\newcommand{\ttt}{\texttt}

\title{Hardware Acceleration for A Singuarly Valuable Decomposition \\ TYRION}

\author{
	Chae Jubb \\ \ttt{ecj2122@columbia.edu} \\ Columbia University
	\and
	Ruchir Khaitan \\ \ttt{rk2660@columbia.edu} \\ Columbia University
}
\date{}

\begin{document}
\maketitle

\section*{Overview}
We intend to build a hardware accelerator for the singular value decomposition
(SVD) and maybe the randomized SVD. Essentially, this will be a peripheral to
compute basic matrix operations like matrix multiply, transpose, etc. Those
basic steps can then be composed in hardware to compute subportions of a Jacobi
SVD algorithm using the Kogbetliantz method, which is very parallelizable. This
approach was implemented by Ma, Kaye, et al in 2006 \cite{FPGA_SVD_ma}.  We will
also be consulting other supplementary sources to determine the final
implementation \cite{berry2006parallel, halko2011finding}.  We will optimize the
algorithm specifically for the SoCKit by appropriately splitting the workload
between the onboard ARM processor and the FPGA.

\section*{Evaluation}
Our evaluation will involve comparing the performance of our joint
hardware-software implementation against a pure software implementation.  This
implementation will be either obtained or written and serve as the origination
of our porting.

\section*{Project Requirements}
Our project requires only the Cyclone SoCKit board.  We will be utilizing the
ARM processor as well as the FPGA to optimize the performance of the SVD
algorithm.

\section*{Milestones}
\begin{enumerate}
	\item Implement an SVD algorithm in C
	\item Define interface between onboard processor and FPGA
	\item Implement an SVD algorithm split between the FPGA and ARM processor
\end{enumerate}

%Bibliography
{\small
\bibliographystyle{abbrv}
\bibliography{ref}
}

\end{document}
